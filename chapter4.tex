\newpage
\chapter{
نتیجه‌گیری و  پیشنهادات
}

در این پژوهش، به منظور کاهش تزویج متقابل در آرایه‌های آنتن مایکرواستریپ، دو روش اصلی شامل ساختار زمین ناقص (DGS) و فراماده‌ها به طور جامع مورد بررسی و تحلیل قرار گرفت. با توجه به نتایج به دست آمده از شبیه‌سازی‌ها و مطالعات تجربی، هر یک از این روش‌ها کارایی قابل توجهی در کاهش تزویج متقابل نشان دادند. به طور مشخص، استفاده از ساختار DGS موفق به کاهش ۱۲ دسی‌بل، و به کارگیری ترکیب فراماده‌ با DGS منجر به کاهش ۱۸ و ۳۲ دسی‌بل در تزویج متقابل شد.

نتایج این پژوهش نشان داد که انتخاب روش مناسب به نیازهای خاص پروژه بستگی دارد. به عنوان مثال، ساختار DGS با پیاده‌سازی ساده‌تر و هزینه کمتر، گزینه مناسبی برای کاربردهای متداول است، در حالی که فراماده‌ها با وجود پیچیدگی طراحی بیشتر، کاهش تزویج بالاتری را ارائه می‌دهند و برای سیستم‌های با حساسیت بالا مناسب هستند. علاوه بر این، تأثیر این روش‌ها بر سایر پارامترهای آنتن از جمله پهنای باند، بهره و الگوی تشعشعی نیز مثبت ارزیابی شد.


\section{
	پیشنهادات برای تحقیقات آینده
}

با توجه به یافته‌های این پژوهش، پیشنهادات زیر برای تحقیقات آینده ارائه می‌شود:
\begin{enumerate}
	\item {
	.بررسی تأثیر ترکیب ساختارهای فراماده و DGS بر عملکرد آنتن با تغییر فاصله بین المان‌های آرایه.
	}
	\item {
	بهینه‌سازی چندهدفه: توسعه الگوریتم‌های بهینه‌سازی برای دستیابی همزمان به چندین هدف شامل کاهش تزویج، افزایش پهنای باند و بهبود بهره.
	}
	\item {
	 کاربرد در فرکانس‌های بالاتر: ارزیابی عملکرد این روش‌ها در فرکانس های بالاتر برای مثال کاربردهای مخابراتی نسل پنجم و ششم.
	}
	\item {
	استفاده از مواد پیشرفته: به کارگیری مواد با خواص الکترومغناطیسی قابل کنترل برای دستیابی به کاهش تزویج گسترده‌تر.
	}
	\item {
	پیاده‌سازی عملی و آزمایش‌های میدانی: ساخت نمونه‌های عملی و انجام آزمایش‌های میدانی برای اعتبارسنجی نتایج شبیه‌سازی.
	}
	\item {
	 توسعه مدل‌های تحلیلی: ارائه مدل‌های تحلیلی دقیق‌تر برای پیش‌بینی عملکرد این ساختارها بدون نیاز به شبیه‌سازی‌های زمان‌بر.
	}
\end{enumerate}

این پژوهش گامی مؤثر در جهت بهبود عملکرد آرایه‌های آنتن مایکرواستریپ بود و امید است نتایج آن زمینه‌ساز تحقیقات آینده در این حوزه شود. با توجه به رشد سریع فناوری‌های ارتباطی، توسعه روش‌های کارآمد برای کاهش تزویج متقابل همچنان به عنوان یک زمینه تحقیقاتی فعال و پرچالش باقی خواهد ماند.

