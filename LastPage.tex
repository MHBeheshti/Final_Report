\newpage
\thispagestyle{empty}
\begin{latin}
\textbf{Abstract:}
One-bit compressive sensing (CS) is an advanced version of compressed sensing in which the sparse signal of interest can be recovered from extremely quantized measurements. Namely, only the sign of each measurement is available to us. Many signals of interest are sparsely represented only in a redundant dictionary. A strong line of research has addressed conventional compressed sensing in this signal model including its extension to one-bit measurements. Although, one-bit CS suffers from the extremely large number of required measurements to achieve a predefined reconstruction error level. Adaptive sampling acts on acquired samples to trace the signal in an efficient way. Adaptive sampling have been employed in conventionally sparse one-bit CS. In this work, adaptive sampling is generalized to dictionary sparse signals in binary measurements. A multidimensional threshold is used to incorporate current estimate of the signal and prevent less informative sampling schemes. This strategy substantially reduces the required number of measurements for exact recovery. Recent geometric concepts in high dimensional geometry like random hyperplane tessellation and Gaussian width are used in the process of proofs. It is shown through rigorous and numerical analysis that the proposed algorithm considerably outperforms state of the art approaches. Further, it reaches exponential error decay in terms of quantized measurements quantity.


\textbf{keywords}: 
One-bit, Dictionary sparse, Adaptive measurement, High dimensional geometry.
\end{latin}
\newpage
\thispagestyle{empty}
\vspace*{-28mm}

\centerline{\includegraphics[scale=1.3]{./Images/general/IUST_logo_en.png}}
\begin{latin}
\begin{center}
\large
\vspace{-1mm}
\textbf{School of Electrical Engineering}
\\[3cm]
\textbf{Adaptive Dictionary Sparse Signal Recovery Using Binary Measurements}
\\[1.5cm]
A Thesis Submitted in Partial Fulfillment of the Requirement for the 
\\[0.5cm]
Degree of Master of Science in Telecommunication systems 
\\[1cm]
By: 
\\[0.5cm]
\textbf{Hossein Beheshti}
\\[1cm]
Supervisor:
\\[0.5cm]
Dr. Farzan Haddadi
\\[1cm]
June 2018 

\end{center}
\end{latin}