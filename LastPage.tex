\newpage
\thispagestyle{empty}
\begin{latin}
\textbf{Abstract:}
Microstrip antennas are widely used in communication systems due to their simple fabrication and small size. However, their use in arrays leads to mutual coupling between elements, which results in consequences like gain reduction and distortion of the radiation pattern. In this study, to investigate and reduce this effect, after introducing microstrip and array antennas, we first reproduced the classical model, which included two rectangular patches at different distances. Then, based on the Carver article, at a distance of a quarter wavelength at a frequency of 1.41 GHz, a DGS (Defected Ground Structure) was designed and applied to improve mutual coupling. Furthermore, a metasurface was placed between the two antennas at a distance from the antenna's plane to create a stopband for surface waves, further reducing mutual coupling. Simulation results in HFSS showed that the combination of the DGS and metasurface led to a reduction of about 32 dB in
$\vert\bm{S}_{21}\vert$
, while 
$\vert\bm{S}_{11}\vert$
and the primary radiation pattern remained almost constant. These results demonstrate the effectiveness of meta-structures in improving the performance of microstrip arrays.



\textbf{keywords}: 
Microstrip antenna, Mutual coupling, Coupling reduction, Patch antennas, Micropatch antennas
\end{latin}
\newpage
\newpage
\thispagestyle{empty}
\vspace*{-28mm}

\centerline{\includegraphics[scale=0.5]{./Images/general/logo_en.png}}
\begin{latin}
\begin{center}
\large
\vspace{-1mm}
\textbf{School of Engineering}
\\[3cm]
\textbf{Mutual coupling reduction between two microstrip patch antenna}
\\[1.5cm]
Undergraduate Final Project Report
\\[4cm]
By: 
\\[0.5cm]
Mohammad Hassan Beheshti
\\[1cm]
Supervisor:
\\[0.5cm]
Dr. Hamid Reza Hassani
\\[2cm]
September 2025

\end{center}
\end{latin}