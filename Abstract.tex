\chapter*{چکیده}
\thispagestyle{empty}
\section*{چکیده}
حسگری فشرده تک بیتی یک نسخه از حسگری فشرده است که در آن سیگنال تنک مورد بررسی از نمونه‌های کوانتیزه شده‌ی تک بیتی، قابل بازیابی است. به عبارت دیگر در این حالت تنها علامت هر اندازه‌گیری در اختیار ما قرار دارد. در پردازش سیگنال‌های دیجیتال بسیاری از سیگنال‌ها در یک دیکشنری افزونه تنک هستند. بخش گسترده‌ای از تحقیقات در حسگری فشرده‌ی سنتی به بررسی این مدل از سیگنال‌ها پرداخته و در ادامه، بازیابی سیگنال‌های دیکشنری-تنک به حسگری فشرده تک بیتی گسترش داده شده است. با این اوصاف، یکی از مشکلات اصلی در حسگری فشرده تک بیتی، تعداد نمونه‌‌های بسیار زیاد جهت دسترسی به خطای بازیابی مطلوب است. با استفاده از نمونه‌برداری وفقی، می‌توان سیگنال 	را دنبال نمود. این الگوی نمونه‌برداری در حسگری فشرده‌ی تک بیتی سنتی اعمال شده است. در این پایان‌نامه، نمونه‌برداری وفقی و تک بیتی به سیگنال‌های دیکشنری تنک تعمیم یافته است. نکته‌ی کلیدی در روش نمونه‌برداری مورد استفاده در این کار، استفاده از آستانه‌گذاری چند بعدی بوده که از دریافت نمونه با فاصله‌ی بالا از سیگنال مجهول، جلوگیری می‌کند. با استفاده از این استراتژی، تعداد نمونه‌های لازم جهت بازیابی دقیق به شدت کاهش می‌یابد.  تحلیل تئوری و نتایج شبیه‌سازی نشان‌دهنده‌ی تاثیر چشم‌گیر استفاده از الگوریتم ارائه شده در کاهش خطای بازیابی (کاهش نمایی) نسبت به حالت غیر وفقی است. در مقابل به ازای افزایش دقت بازیابی، پیچیدگی نمونه‌برداری افزایش می‌یابد که در برخی از کاربرد‌ها استفاده از الگوریتم پیشنهادی را محدود می‌کند. عملکرد الگوریتم ارائه شده به صورت هندسی توضیح داده شده و  جهت اثبات نتایج از مفاهیم جدید در هندسه‌ی ابعاد بالا، استفاده شده است.

\textbf{
کلمات کلیدی:
}
حسگری فشرده، نمونه‌برداری تک-بیتی، سیگنال دیکشنری-تنک، هندسه‌ی ابعاد بالا

