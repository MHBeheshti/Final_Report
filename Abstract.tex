\chapter*{چکیده}
\thispagestyle{empty}
\section*{چکیده}

آنتن‌های میکرواستریپ به دلیل سادگی ساخت و ابعاد کوچک کاربرد وسیعی در سامانه‌های مخابراتی دارند. با این حال، استفاده از آن‌ها در قالب آرایه موجب تزویج متقابل میان عناصر می‌شود که پیامدهایی همچون افت بهره و اعوجاج الگوی تشعشعی را به همراه دارد. در این پژوهش، برای بررسی و کاهش این اثر پس از معرفی آنتن مایکرواستریپ و آنتن آرایه ای، ابتدا مدل کلاسیک 
\lr{Carver \& Jedlicka}\cite{carver1981microstrip}
 شامل دو پچ مستطیلی در فواصل مختلف بازتولید گردید. سپس با پایه ی مقاله ی کارور در فاصله‌ی یک‌چهارم طول‌موج در فرکانس
\lr{1.41 GHz}
   و با الهام از مقاله
\cite{hajilou2012mutual}،
     یک ساختار
\lr{DGS}
       طراحی و اعمال شد. در ادامه، یک متاسرفیس با فاصله از صفحه ی آنتن، میان دو آنتن قرار گرفت تا با ایجاد باند توقف امواج سطحی، کوپلینگ بیشتر کاهش یابد. نتایج شبیه‌سازی در 
       \lr{HFSS}
        نشان داد که ترکیب
\lr{DGS}\LTRfootnote{Defected ground structure}
          و متاسرفیس موجب کاهش تا حدود 32 دسی‌بل در 
$\vert\bm{S}_{21}\vert$
در حالی که 
$\vert\bm{S}_{11}\vert$
 و الگوی اصلی تشعشعی تقریباً ثابت ماندند. این نتایج نشان‌دهنده‌ی کارآمدی فراساختارها در بهبود عملکرد آرایه‌های میکرواستریپ است.
 
 
\textbf{کلمات کلیدی:}
آنتن میکرواستریپ، کوپلینگ متقابل، کاهش کوپلینگ، آنتن‌های پچ، آنتن‌های میکروپچ

